\section{Conclusão}
No artigo foi descrito o desenvolvimento e a implementação do L-System no Unity atráves do asset denominado de LSystem2Unity, no qual dá aos desenvolvedores a possibilidade de aplicar o L-Sistemas em seus projetos com maior facilidade, desde criação de regras para o l-sistema, quanto ao interpretador TurtleAgent.

Durante o desenvolvimento, percebeu-se que é simples de utilizar, podendo protótipar rapidamente uma geração, definindo somente as regras de reescrita do LSystem e fazendo a ligação dos eventos no TurtleAgent com os métodos a serem utilizados pelo desenvolvedor.

A ferramenta ainda possui certas limitações referente a quantidade de tipos de regras a serem utilizados, possuindo somente um sistema implementado, diminuindo as opções de diversificação aos desenvolvedores.

\section{Trabalhos Futuros}
É planejado dar a ferramenta mais estruturas de regras, dando aos desenvolvedores uma maior variedade de opções nos desenvolvimentos de assets em seus jogos. Trazer mais funcionalidades ao editor do Unity, melhorando a utilização da ferramenta. Um ponto essencial é melhorar a performance, diminuindo os recursos de memória e demora de processamento das gerações em grande cadeias de caracteres.

A possibilidade da implementação para a produção de estruturas no editor, deixando os designers criarem conteúdos para os jogos, podendo optimizar e editar os elementos já produzidos.