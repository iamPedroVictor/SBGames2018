\section{Conclusão}
No artigo foi descrito o desenvolvimento e a implementação do L-System no Unity atráves do asset denominado de LSystem2Unity, no qual dá aos desenvolvedores a possibilidade de aplicar o L-Sistemas em seus projetos com maior facilidade, desde criação de regras para o l-sistema, quanto ao interpretador TurtleAgent.

Durante sua aplicação, pode observar que ele é simples de utilizar e que consegue gerar estruturas rapidamente. Pode-se ver que pode criar fractais quanto cenários simples, a ferramenta conseguiu ter um processo eficiente durante as execuções das gerações. Ela apresenta limitações como somente um sistema de regras do L-System implementada, não dando assim, tanta opção de diversificação aos desenvolvedores, tendo assim que recorrer algoritmos mais complexos para ter mais variedades aos seus projetos.

A ferramenta consegue atingir seu objetivo de dar ao desenvolvedor a utilização do sistema de reescrita de estruturas, dependendo somente do desenvolvedor conecte seus algoritmos para construção de seus elementos usando o TurtleAgent.

\section{Trabalhos Futuros}
É planejado dar a ferramenta mais estruturas de regras, dando aos desenvolvedores uma maior variedade de opções nos desenvolvimentos de assets em seus jogos. Trazer mais funcionalidades ao editor do Unity, melhorando a utilização da ferramenta. Um ponto essencial é melhorar a performance, diminuindo os recursos de memória e demora de processamento das gerações em grande cadeias de caracteres.

A possibilidade da implementação para a produção de estruturas no editor, deixando os designers criarem conteúdos para os jogos, podendo optimizar e editar os elementos já produzidos.