\section{Desenvolvimento da ferramenta}

%Para o desenvolvimento da ferramenta foi observado a presença de duas etapas em torno do L-System, a primeira que é a parte de reescrita da estrutura inicial seguindo as regras de produção pre-definidas, a segunda etapa gira em torno da interpretação da estrutura final pelo algoritmo da tartaruga.

%A segunda etapa por sua vez é mais abstrata, pois cabe ao desenvolvedor que for utilizar a ferramenta possa decidir como vai ser os passos da tartaruga para cada elemento dentro da estrutura final. Com isso foi criado duas classes MonoBehaviour: LSystem e TurtleAgent.

Para o desenvolvimento da ferramenta, observou-se a presença de duas etapas, a primeira com o processo de reescrita da estrutura inicial com regras pre-definidas e a uma segunda com a interpretação da estrutura final fornecida pela primeira etapa.

\subsection{LSystem}
%Essa classe faz a implementação da primeira etapa, a de reescrever a estrutura por n gerações, nela o usuário irá poder definir a string inicial, números de gerações, tempo de espera entre as gerações da estrutura, se deve chamar os Turtle Agents no final de cada geração e por ultimo mantem as regras de reescrita. Para o armazenamento das regras, foi criado um scriptableobject Ruleset, que armazena uma lista de objetos da classe Rule, por ser um scriptableobject pode se criar um único dicionario e direcionar ele para diversos LSystem que estão presentes na cena, não tendo a necessidade de repetir a mesma regra diversas vezes.

Esse componente faz o processo da primeira etapa, o de reescrever a estrutura inicial através de suas interações, o componente fornece ao desenvolvedor a possibilidade de realizar as configurações básicas do sistema, como o axioma inicial, numero de gerações, setar as lista de regras de reescrita. Para o armazenamo das regras foi criado o objeto Ruleset, um scriptableobject que possui a lista de elementos chaves-valor.

Após terminar a quantidade de gerações máximas definidas, a classe chama os objetos do tipo TurtleAgent passando a string final gerada, assim iniciando a segunda etapa do sistema que é de interpretação da estrutura.

\subsection{TurtleAgent}
A segunda etapa irá tratar da interpretação dos elementos presentes na estrutura final, para isso ele mantem uma serie de regras que possuem duas propriedades: Symbol e Action, um char e UnityEvent respectivamente. Por utilizar o UnityEvent se cria possibilidade de passar as funções construídas e presentes nos GameObjects da cena, mantendo uma forma de deixar mais genérica a interpretação das estruturas.