\section{Introdução}
Com a facilidade no desenvolvimento de jogos digitais criada nos últimos anos e com um mercado que ganhou bastante valor, tornou-se comum o surgimento de estúdios independentes, normalmente com equipes pequenas com poucos recursos e habilidades diversificadas em seus times, para poderem entregar seus projetos com conteúdos diversificados em pouco tempo a maioria demostrou interesse por técnicas de geração procedural de conteúdo (PCG).

O Sistema de Lindenmayer ou L-System\cite{Prusinkiewicz} se tornou umas das técnicas bastante popular hoje em dia, no qual foi originalmente proposto por Aristid Lindenmayer para a modelagem de plantas, mas sua formula é capaz de ser utilizada em diversas aplicações, como a criação de fractais, geração de musicas em jogos digitais \cite{Fridenfalk} , modelagem de plantas para jogos, criação de labirintos, etc. Graças a essa diversificação, o sistema acaba sendo bastante atraente para produzir essas estruturas através de gramaticas para reescrita.

%O artigo trata-se do desenvolvimento de uma ferramenta para a engine Unity, com o objetivo de agilizar o desenvolvimento em jogos que utilizam o l-sistema, com isso sendo capaz de fácil aplicação para que desenvolvedores utilizem esse sistema de reescrita de estruturas em seus projetos, sendo capazes de reaproveitar as mesmas regras de substituição em outras cenas do jogo ou até mesmo em outros projetos.

Ao observar esses elementos, o artigo apresenta o desenvolvimento de um plugin para a engine Unity com o objetivo de agilizar a aplicação de L-Sistemas em jogos digitais, sendo capaz de fácil aplicação do sistema de reescrita de estruturas em diversos projetos.

O artigo esta organizado da seguinte forma: Na seção 02 é apresentado alguns trabalhos que utilizam o l-sistema; a seção 03 apresenta os conceitos que se encaixam ao projeto; a seção 04 mostra o desenvolvimento da ferramenta e suas etapas de execução; a seção 05 demostra a ferramenta construída e suas interfaces; na seção 06 é apresentado exemplos construídos com a ferramenta mostrando suas regras de produção, a seção 07 mostra o que foi atingido com a ferramenta, por ultimo, a seção 08 discuti alguns pontos a serem construídos e desenvolvidos na ferramenta em interações futuras.