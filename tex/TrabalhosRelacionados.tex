\section{Trabalhos Relacionados}
Em \cite{etchebeheresystems} o l-sistema foi utilizado para a geração de labirintos usando a variação OL-System que permite a obtenção de diferentes resultados para cada vez que um cenário for criado.

Em \cite{fridenfalk2015application} traz a geração de cenas virtuais 3D em tempo de execução, com a adição de um física rudimentar com gravidade e a capacidade de detectar colisões. A aplicação da a possibilidade de configurar diversos modos de execução, com duas principais, o de reprodução e de design.

Já \cite{santos1desenvolvimento} demostra a criação do L-System View para a geração de fractais com o l-sistema, implementando um interpretador de texto, aonde vão os comandos permitindo o usuário de modelar os fractais a serem produzidos, um interpretador que gera um l-sistema correspondente ao fractal e um desenho do fractal modelado, por fim apresenta um painel gráfico que mostra a interpretação gráfica do fractal modelado.